\documentclass{article}

\begin{document}

\section*{Reviewer \#1}

\subsection*{General Concern}
{\em However it seems to me that the potential of these methods is diluted in the task they are used for. What is the use of predicting age from MEG signals when it can be so easily predicted by much simpler metrics? If practical applications is not the goal, but the goal is a more scientific understanding of speech development, then I am not sure how a coarse metric like "age $>=10$" relates to it. What the algorithm is detecting brain features that are different across children of age $>=10$ or $<10$. I am not convinced that the features detected are relevant to speech development.}


\subsection*{Suggested Fixes}
{\em I suggest that the authors perform some of these checks in order to make their conclusions more impactful:}
  \begin{enumerate}
  \item{try to predict a closer measure of language development if such a measure was recorded}
  \item{try to predict the actual production task (which of the three production tasks each trial corresponds to) instead of age to check their algorithms: the artificial data that the authors simulate by maximizing the classifier output shows that the time dimension is not effectively utilized by the network. It seems to me from the manuscript that the networks are rather utilizing frequency band information that is different in older children from younger children. In order to see if the temporal filtering detects temporal patterns (relevant to stimulus onset and offset etc.), training the networks to distinguish between trials might showcase the power of these networks at learning temporal MEG patterns. Note: this could have happened in the BCI tasks that were used, but these results were not expanded on.}
  \item{if the reason why the networks have a good accuracy is that they are sensitive to the difference in frequency bands, then it's important to know if these frequency bands are related to the tasks or not. If possible, the analysis can be repeated with the portion of the data before (and after) the actual onset of the spoken words. If the accuracy persists to be above chance, then maybe what is being classified is a baseline activity and not something about speech production.}
    \item{figure 2 shows that the most relevant channels are typically the ones on the outer sides of the helmet. There is a hot spot on the upper left which could be related to left inferior frontal cortex, but of course it's hard to tell with MEG. One thing I was wondering about is how important head size is. Is it (partially) what is being classified? Head size, or movement, could easily be correlated to age. Can the authors think of additional tests to run that would clarify this issue?}
  \end{enumerate}

\subsection*{Clarity Concerns}



\end{document}